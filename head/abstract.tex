%!TEX root = ../my_thesis.tex

\cleardoublepage
\chapter*{Abstract}
\vskip1em

A software-defined radio is a radio communication system where components
traditionally implemented in hardware are instead implemented by means of
software. With the growing number of complex digital communication standards and
the general purpose processors increasing power, it becomes interesting to
trade the energy efficiency of the dedicated architectures for the flexibility
and the reduced time to market on general purpose processors.

Even if the resulting implementation of a signal processing is made on an
application-specific integrated circuit, the software version of this processing
is necessary to evaluate and verify the correct properties of the functionality.
This is generally the role of the simulation. Simulations are often expensive in
terms of computational time. To evaluate the global performance of a
communication system can require from few days to few weeks.

In this context, this thesis proposes to study the most time consuming
algorithms in today's digital communication chains. These algorithms often are
the channel decoders located on the receivers. The role of the channel coding is
to improve the error resilience of the system. Indeed, errors can occur at the
channel level during the transmission between the transmitter and the receiver.
Three main channel coding families are then presented: the LDPC codes, the polar
codes and the turbo codes. These three code families are used in most of the
current digital communication standards like the Wi-Fi, the Ethernet, the 3G, 4G
and 5G mobile networks, the digital television, etc. The resulting decoders
offer the best compromise between error resistance and decoding speed known to
date. Each of these families comes with specific decoding algorithms. One of the
main challenge of this thesis is to propose optimized software implementations
for each of them. Specific efficient implementations are proposed as well as
more general optimization strategies. The idea is to extract the generic
optimization strategies from a representative subset of decoders.

The last part of the thesis focuses on the implementation of a complete digital
communication system in software. Thanks to the efficient decoding
implementations proposed before, a full transceiver, compatible with the DVB-S2
standard, is implemented. This standard is typically used for broadcasting
multimedia contents via satellite. To this purpose, an embedded domain specific
language targeting the software-defined radio is introduced. The main objective
of this language is to take advantage of the parallel architecture of the
current general purpose processors. The results show that the system achieves
sufficient throughputs to be deployed in real-world conditions.

These contributions have been made in a dynamic of openness, sharing and
reusability, it results in an open source library named AFF3CT for A Fast
Forward Error Correction Toolbox. Thus, all the results proposed in this thesis
can easily be reproduced and extended. This philosophy is detailed in a specific
chapter of the thesis manuscript.

\vskip0.5cm
\emph{Keywords:} Software-Defined Radio, Functional Simulation, Error
                 Correcting Codes, Software Implementation, Optimization,
                 Parallelization, Open Source Code
