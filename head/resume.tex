%!TEX root = ../my_thesis.tex

\cleardoublepage
\addcontentsline{toc}{chapter}{Abstracts}
\chapter*{Résumé}
\vskip1em

La radio logicielle consiste à remplacer les traitements numériques auparavant
réalisés à l'aide de matériels dédiés par des implémentations logicielles. De
plus en plus, avec l'émergence de nouveaux standards de communication complexes
et la puissance de calcul grandissante des processeurs généralistes, il devient
intéressant d'échanger l'efficacité énergétique des architectures dédiés par
la souplesse et la facilité d'implémentation sur processeurs généralistes.

Même si l'implémentation d'un traitement numérique est finalement faite sur une
puce dédiée, la version logicielle de ce traitement est nécessaire afin de
vérifier les bonnes propriétés de l'algorithme. Cela est généralement le rôle de
la simulation: les concepteurs de systèmes de communication numériques évaluent
d'abord les propriétés des algorithmes de traitement du signal dans des
simulations logicielles. Ces dernières sont souvent coûteuses en temps de
calcul. Il n'est pas rare de devoir attendre plusieurs jours voire plusieurs
semaines pour évaluer les performances globales d'un système en logiciel.

Dans ce contexte, cette thèse propose d'étudier les algorithmes les plus
coûteux en temps de calcul dans les chaînes de communication numériques
actuelles. Ces algorithmes sont le plus souvent les décodeurs de codage canal
que l'on trouve dans les récepteurs. Le rôle du codage canal est d’accroître la
résistance aux erreurs qui peuvent apparaître lorsque l'information transite
entre l’émetteur et le récepteur. Trois grandes familles de codage canal sont
alors présentées, à savoir les codes LDPC, les codes polaires et les turbo
codes. Ces trois familles de code sont utilisées dans la plupart des standards
de communication actuels comme le Wi-Fi, l’Ethernet, les réseaux mobiles 3G, 4G
et 5G, la télévision numérique, etc. Les décodeurs qui en découlent proposent le
meilleur compromis résistance aux erreurs sur vitesse de décodage que l'on
connaisse à ce jour. Chacune de ces familles vient avec des algorithmes de
décodage différents. Un des enjeux principal de cette thèse est de proposer
des implémentations logicielles optimisées pour chacun d'entre eux. Des réponses
sont apportées au cas par cas et des stratégies d'optimisation plus générales
sont aussi discutées. L'idée étant d'abstraire les stratégies d'optimisation
possibles en étudiant un sous-ensemble de décodeurs représentatif.

Enfin, la dernière partie de cette thèse s'attache à la mise en œuvre d'un
système de communication numérique complet à l'aide de la radio logicielle. Fort
des implémentations rapides de décodeurs proposées précédemment, un émetteur et
un récepteur compatibles avec le standard DVB-S2 sont implémentés. Ce standard
est typiquement utilisé pour la diffusion de contenu multimédia par satellite. À
cette occasion, un langage dédié à la radio logicielle a été développé pour
tirer parti de l'architecture parallèle des processeurs généralistes actuels.
Les résultats démontrent que le système atteint des débits suffisants pour être
déployé en conditions réelles.

Ces contributions ont été réalisées dans une dynamique d'ouverture et de
partage, il en résulte une bibliothèque à code source ouvert nommée AFF3CT pour
\emph{A Fast Forward Error Correction Toolbox}. Ainsi, les résultats proposés
dans cette thèse peuvent aisément être reproduits. Cette philosophie est
détaillée dans un chapitre du manuscrit.

\vskip0.5cm
\emph{Mots clefs :} Radio logicielle, Simulation, Codes correcteurs d'erreurs,
                    Implémentation logicielle, Optimisation, Parallélisation,
                    Code source ouvert
