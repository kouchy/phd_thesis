\documentclass[a4paper, 11pt]{article}
\usepackage[utf8]{inputenc}
\usepackage[T1]{fontenc}
\usepackage[english,french]{babel}
\usepackage[top=1.25cm,bottom=1.25cm,right=1.5cm,left=1.5cm]{geometry}
\usepackage{eulervm}
\usepackage{tikz}
\usetikzlibrary{calc,positioning}
\usepackage{color}
\definecolor{bleuUni}{RGB}{0, 157, 224}
\begin{document}

\thispagestyle{empty}
\selectlanguage{french}

\noindent
{\large \textbf{\og Méthodes d'optimisation et de parallélisation pour la radio
                logicielle \fg}}

\paragraph{Résumé}

Cette thèse propose d'étudier les algorithmes les plus coûteux en temps de
calcul dans les chaînes de communication numériques actuelles. Ces algorithmes
sont le plus souvent présents dans des décodeurs de codes correcteurs d'erreurs
au niveau récepteur. Le rôle du codage canal est d'accroître la robustesse vis à
vis des erreurs qui peuvent apparaître lorsque l'information transite au travers
d'un canal de transmission. Trois grandes familles de codes correcteurs
d'erreurs sont étudiées dans nos travaux, à savoir les codes LDPC, les codes
polaires et les turbo codes. Ces trois familles de codes sont présentes dans la
plupart des standards de communication actuels comme le Wi-Fi, l'Ethernet, les
réseaux mobiles 3G, 4G et 5G, la télévision numérique, etc. Les décodeurs qui en
découlent proposent le meilleur compromis entre la résistance aux erreurs et la
vitesse de décodage. Chacune de ces familles repose sur des algorithmes de
décodage spécifiques. Un des enjeux principal de cette thèse est de proposer des
implémentations logicielles optimisées pour chacune des trois familles. Des
réponses sont apportées de façon spécifique puis des stratégies d'optimisation
plus générales sont discutées (vectorisation et parallélisation). L'idée est
d'abstraire des stratégies d'optimisation possibles en étudiant un sous-ensemble
représentatif de décodeurs.

Une autre contribution importante de cette thèse est la mise en œuvre d'un
système de communications numériques complet à l'aide de la radio logicielle. En
s'appuyant sur les implémentations rapides de décodeurs proposées, un émetteur
et un récepteur compatibles avec le standard DVB-S2 sont implémentés. Ce
standard est typiquement utilisé pour la diffusion de contenu multimédia par
satellite. À cette occasion, un langage dédié à la radio logicielle est
développé pour tirer parti de l'architecture parallèle des processeurs
généralistes actuels. Le système atteint des débits suffisants pour être déployé
en condition opérationnelle.

\vskip0.5cm
\emph{Mots clefs :} Radio logicielle, Simulation fonctionnelle, Codes
                    correcteurs d'erreurs, Implémentation logicielle,
                    Optimisation, Parallélisation, Code source ouvert

\vskip0.5cm
{\color{bleuUni}\hrule}
\vskip0.5cm

\selectlanguage{english}

\noindent
{\large \textbf{``Optimization and Parallelization Methods for the
                Software-Defined Radio''}}

\paragraph{Abstract}

This thesis proposes to study the most time consuming algorithms in today's
digital communication chains. These algorithms often are the channel decoders
located on the receivers. The role of the channel coding is to improve the error
resilience of the system. Indeed, errors can occur at the channel level during
the transmission between the transmitter and the receiver. Three main channel
coding families are then presented: the LDPC codes, the polar codes and the
turbo codes. These three code families are used in most of the current digital
communication standards like the Wi-Fi, the Ethernet, the 3G, 4G and 5G mobile
networks, the digital television, etc. The resulting decoders offer the best
compromise between error resistance and decoding speed known to date. Each of
these families comes with specific decoding algorithms. One of the main
challenge of this thesis is to propose optimized software implementations for
each of them. Specific efficient implementations are proposed as well as more
general optimization strategies (vectorization and parallelization). The idea is
to extract the generic strategies from a representative subset of decoders.

Another important contribution of this thesis is the implementation of a
complete digital communication system in software. Thanks to the efficient
decoding implementations proposed before, a full transceiver, compatible with
the DVB-S2 standard, is implemented. This standard is typically used for
broadcasting multimedia contents via satellite. To this purpose, an embedded
domain specific language targeting the software-defined radio is introduced. The
main objective of this language is to take advantage of the parallel
architecture of the current general purpose processors. The results show that
the system achieves sufficient throughputs to be deployed in real-world
conditions.

\vskip0.5cm
\emph{Keywords:} Software-Defined Radio, Functional Simulation, Error
                 Correcting Codes, Software Implementation, Optimization,
                 Parallelization, Open Source Code


\begin{tikzpicture}[overlay,remember picture]
  \node[opacity=1.0] at (0.9,-1.75) {\includegraphics[height=3.5cm]{fig/qr_code}};

  \node[                     ] (ims) at (15.8,-2.50) {\includegraphics[height=1.2cm]{../head/pic/ims}};
  \node[left = 0.1cm of ims  ] (inria)               {\includegraphics[height=1.5cm]{../head/pic/inr_logo_rouge}};
  \node[left = 0.1cm of inria] (ub)                  {\includegraphics[height=1.5cm]{../head/pic/ub}};
\end{tikzpicture}

\end{document}
