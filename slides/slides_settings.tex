%!TEX root = ./slides.tex

% Packages --------------------------------------------------------------------
\usepackage[T1]{fontenc}
\usepackage[utf8]{inputenc}
% \usepackage[frenchb]{babel}
\usepackage[english]{babel}
\usepackage{eulervm}
\usepackage{etoolbox,refcount}
\usepackage[normalem]{ulem} % strikeout with \sout{}
\usepackage{booktabs}
\usepackage{multirow}
\usepackage{multicol}
\usepackage{makecell}
\usepackage{pgfplots}
\usepackage{tikz}
\usepackage{circuitikz}
\usepackage{tikz-timing} % package pour les chronogrammes
\usepackage{caption}
\usepackage{graphicx}
\usepackage[export]{adjustbox} % align in includegraphics
\usepackage{stmaryrd} % llbracket
\usepackage{amsmath}
\usepackage{amssymb}
\usepackage{marvosym} % grosse fleche
\usepackage{calc}
\usepackage[english,onelanguage,ruled,linesnumbered,vlined]{algorithm2e}
\usepackage{bm}
\usepackage{pifont}
\usepackage{minted} % beautiful source code
\usepackage{tcolorbox}
\usepackage{etoolbox}
\usepackage[
  labelnumber  = true,
  backend      = biber,
  % style        = alphabetic,
  % citestyle    = alphabetic,
  maxnames     = 6,
  defernumbers = true,
  isbn         = false,
  doi          = true,
  url          = true,
  backref      = true]{biblatex}
\addbibresource{../tail/bibliography.bib}

% Parameters ------------------------------------------------------------------
% \renewcommand{\scriptsize}{\fontsize{5}{6}}
\renewcommand{\footnotesize}{\fontsize{7}{8.4}}
% % \renewcommand{\footnotesize}{\scriptsize}

\pgfplotsset{compat=newest}
\usepgfplotslibrary{groupplots}
\usetikzlibrary{matrix, positioning, fit, patterns, shapes, arrows, shapes.multipart, decorations.pathmorphing, calc}
\tikzset{
    invisible/.style={opacity=0},
    visible on/.style={alt={#1{}{invisible}}},
    alt/.code args={<#1>#2#3}{%
      \alt<#1>{\pgfkeysalso{#2}}{\pgfkeysalso{#3}} % \pgfkeysalso doesn't change the path
    },
    hatch distance/.store in=\hatchdistance,
    hatch distance=7pt,
    hatch thickness/.store in=\hatchthickness,
    hatch thickness=0.5pt,
}
% \SetAlFnt{\footnotesize}
\AtBeginEnvironment{frame}{\setcounter{footnote}{0}}

% package 'minted'
\setminted{bgcolor=marronUni,fontsize=\fontsize{8}{9.2},tabsize=4,numbers=left,framesep=0pt,numbersep=2pt,xleftmargin=3.5mm,xrightmargin=3.5mm,highlightcolor=marronUni!90} % package 'minted'
\usemintedstyle{monokai} % friendly fruity
% https://tex.stackexchange.com/questions/252263/alignment-of-minted-line-numbers
% \newlength{\mintednumbersep}
% \AtBeginDocument{%
%   \sbox0{\tiny00}%
%   \setlength\mintednumbersep{\parindent}%
%   \addtolength\mintednumbersep{-\wd0}%
% }
\BeforeBeginEnvironment{minted}{\begin{tcolorbox}[colback=marronUni, colframe=black, arc=0.5mm, boxsep=0mm, boxrule=0.0mm, left=0mm, right=0mm, top=0mm, bottom=0mm]}
\AfterEndEnvironment{minted}{\end{tcolorbox}}
\renewcommand{\theFancyVerbLine}{\textcolor[rgb]{1,1,1}{\tiny{\arabic{FancyVerbLine}}}}
% \renewcommand{\theFancyVerbLine}{\sffamily\textcolor[rgb]{1,1,1}{\footnotesize\oldstylenums{\arabic{FancyVerbLine}}}}

% new shapes for pgfplot
\makeatletter
\pgfdeclarepatternformonly[\hatchdistance,\hatchthickness]{flexible hatch north east}
{\pgfqpoint{0pt}{0pt}}
{\pgfqpoint{\hatchdistance}{\hatchdistance}}
{\pgfpoint{\hatchdistance-1pt}{\hatchdistance-1pt}}%
{
    \pgfsetcolor{\tikz@pattern@color}
    \pgfsetlinewidth{\hatchthickness}
    \pgfpathmoveto{\pgfqpoint{0pt}{0pt}}
    \pgfpathlineto{\pgfqpoint{\hatchdistance}{\hatchdistance}}
    \pgfusepath{stroke}
}
\makeatletter
\pgfdeclarepatternformonly[\hatchdistance,\hatchthickness]{flexible hatch north west}
{\pgfqpoint{0pt}{0pt}}
{\pgfqpoint{\hatchdistance}{\hatchdistance}}
{\pgfpoint{\hatchdistance-1pt}{\hatchdistance-1pt}}%
{
    \pgfsetcolor{\tikz@pattern@color}
    \pgfsetlinewidth{\hatchthickness}
    \pgfpathmoveto{\pgfqpoint{\hatchdistance}{0pt}}
    \pgfpathlineto{\pgfqpoint{0pt}{\hatchdistance}}
    \pgfusepath{stroke}
}

\setbeamertemplate{caption}{\raggedright\insertcaption\par}

% biblatex
% separate multi-citations by a comma instead of a semicolon in 'biblatex'
\renewcommand*{\multicitedelim}{\addcomma\addspace}
% \renewcommand\mkbibacro[1]{{\footnotesize\MakeUppercase{#1}}}
\definecolor{bluecite}{HTML}{009DE0}
\setbeamertemplate{bibliography item}{\textcolor{black}{\insertbiblabel}}
\newcommand{\enumcite}[1]{{\scriptsize\cite{#1}\quad \fullcite{#1}}}

% package 'hyperref'
\hypersetup{
  colorlinks,
  allcolors=.,
  citecolor  = bluecite,
  urlcolor=bluecite,
}
% package 'url'
\urlstyle{tt}


\SetKwComment{Comment}{$\triangleright$\ }{} % package 'algorithm2e'
\newcommand\mycommfont[1]{\small\ttfamily\textcolor{Comment}{#1}} % package 'algorithm2e'
\SetCommentSty{mycommfont} % package 'algorithm2e'


% Beamer template -------------------------------------------------------------
\input{../settings/colors}
\usecolortheme[named=bleuUni]{structure}

% Special commands : ddfrac and actionenv and compresslist
\newcommand\ddfrac[2]{\frac{\displaystyle #1}{\displaystyle #2}}

\newenvironment<>{varblock}[2][\textwidth]{%
  \setlength{\textwidth}{#1}
  \begin{actionenv}#3%
    \def\insertblocktitle{#2}%
    \par%
    \usebeamertemplate{block begin}}
  {\par%
    \usebeamertemplate{block end}%
  \end{actionenv}}

\newcommand{\compresslist}{ % Define a command to reduce spacing within itemize/enumerate environments, this is used right after \begin{itemize} or \begin{enumerate}
\setlength{\itemsep}{1pt}
\setlength{\parskip}{0pt}
\setlength{\parsep}{0pt}
}

\newcommand*\circled[1]{\tikz[baseline=(char.base)]{%
      \node[shape=circle,fill=bleuUni,inner sep=2pt] (char) {\textbf{\textcolor{white}{#1}}};}}

\newcommand{\itmsp}[1] {\setlength\itemsep{#1}}
%%%%%%%%%%%%%%%%%%%%%%%%%

%%%%% Beamer
\usepackage[bars]{beamerthemetree} % Beamer theme v 2.2
\mode<presentation>
\newcommand*\oldmacro{}%
\let\oldmacro\insertshorttitle%
\renewcommand*\insertshorttitle{%
 \oldmacro\hspace{0pt plus 1 filll}%
\insertframenumber\,/\,\inserttotalframenumber}
\setbeamertemplate{footline}[frame number]
\setbeamersize{text margin left=10pt,text margin right=10pt}
\setbeamerfont{frametitle}{size=\small}
\setbeamertemplate{frametitle}{ \nointerlineskip %
\begin{beamercolorbox}[wd=\paperwidth,ht=2.2ex,dp=.9ex,left]{frametitle} %
                       \hspace*{2ex}\strut\bfseries\color{bleuUni!15!white}\insertframetitle\strut\par %
\end{beamercolorbox}}

%\setbeamerfont{headline}{size=\footnotesize}

% \usepackage{animate}
% \usepackage{multimedia}
\usetheme{Ilmenau} % Beamer theme v 3.0
\setbeamercolor{section in head/foot}{bg=marronUni}
\useinnertheme{circles} %rectangle bullet points instead of circle ones
\usepackage{beamerthemebars}
\beamertemplatenavigationsymbolsempty
%\setbeamercolor{navigation symbols dimmed}{fg=red!80!black}
%\setbeamercolor{navigation symbols}{fg=red!80!black}
%%%%%%%%%%%%%%%%%%%%%%%%%

\setbeamertemplate{headline}{%
\begin{beamercolorbox}[colsep=1.5pt]{upper separation line head}
\end{beamercolorbox}
\begin{beamercolorbox}{section in head/foot}
    \vskip2pt\insertsectionnavigationhorizontal{\paperwidth}{}{}\vskip2pt
\end{beamercolorbox}%
\begin{beamercolorbox}[ht=10pt]{subsection in head/foot}%
    \vskip2pt\insertsubsectionnavigationhorizontal{\paperwidth}{}{}\vskip2pt
\end{beamercolorbox}%
\begin{beamercolorbox}[colsep=1.5pt]{lower separation line head}
\end{beamercolorbox}
}

\setbeamertemplate{blocks}[rounded][shadow=false]

%%%%% Contents each new sec and subsec
\AtBeginSection[]
{
  %\ifnumcomp{\value{section}}{=}{1}{}{
    \begin{frame}[c]{Plan}
      \centering
      \tableofcontents[
          currentsection,
          hideothersubsections
          %subsectionstyle=show/hide
      ]
    \end{frame}
  %}
}

\AtBeginSubsection[]
{
  \begin{frame}[c]{Plan}
    \tableofcontents[
      currentsection,
      sectionstyle=show/shaded,
      subsectionstyle=show/shaded/hide
    ]
  \end{frame}
}


% MIPP ------------------------------------------------------------------------

\newcommand{\MIPP}{MIPP\xspace}
\newcommand{\longMIPP}{MyIntrinsics++\xspace}
\newcommand{\TSIMD}{T-SIMD\xspace}
\newcommand{\xsimd}{xsimd\xspace}
\newcommand{\simdpp}{simdpp\xspace}
\newcommand{\Vc}{Vc\xspace}
\newcommand{\VCL}{VCL\xspace}
\newcommand{\BoostSIMD}{Boost.SIMD\xspace}
\newcommand{\bSIMD}{bSIMD\xspace}

\newcommand{\Cxx}{\texttt{C++}\xspace}
\newcommand{\Cxy}[1]{\texttt{C++{#1}}\xspace}
\newcommand{\CppUnit}{\texttt{CppUnit}\xspace}

\newcommand{\cmark}{\textcolor{btfGreen}{\ding{51}}}
\newcommand{\xmark}{\textcolor{btfRed}{\ding{55}}}
\newcommand{\Arikan}{Ar{\i}kan\xspace}

\DeclareMathOperator*{\sign}{sign}
\DeclareMathOperator*{\DecoderSC}{DecoderSC}
\DeclareMathOperator*{\hardDecide}{h_d}
