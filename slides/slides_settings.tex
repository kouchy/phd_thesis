%!TEX root = ./slides.tex

% Packages --------------------------------------------------------------------
\usepackage[T1]{fontenc}
\usepackage[utf8]{inputenc}
% \usepackage[frenchb]{babel}
\usepackage[english]{babel}
\usepackage{eulervm}
\usepackage{etoolbox,refcount}
\usepackage[normalem]{ulem} % strikeout with \sout{}
\usepackage{booktabs}
\usepackage{multirow}
\usepackage{multicol}
\usepackage{makecell}
\usepackage{pgfplots}
\usepackage{tikz}
\usepackage{circuitikz}
\usepackage{tikz-timing} % package pour les chronogrammes
\usepackage{caption}
\usepackage{graphicx}
\usepackage[export]{adjustbox} % align in includegraphics
\usepackage{stmaryrd} % llbracket
\usepackage{amsmath}
\usepackage{amssymb}
\usepackage{marvosym} % grosse fleche
\usepackage{calc}
\usepackage[english, onelanguage, ruled, linesnumbered, vlined]{algorithm2e}
\usepackage{bm}

% Parameters ------------------------------------------------------------------
\pgfplotsset{compat=newest}
\usepgfplotslibrary{groupplots}
\usetikzlibrary{matrix, positioning, fit, patterns, shapes, arrows, shapes.multipart, decorations.pathmorphing, calc}
\tikzset{
    invisible/.style={opacity=0},
    visible on/.style={alt={#1{}{invisible}}},
    alt/.code args={<#1>#2#3}{%
      \alt<#1>{\pgfkeysalso{#2}}{\pgfkeysalso{#3}} % \pgfkeysalso doesn't change the path
    },
}
\SetAlFnt{\footnotesize}

% Beamer template -------------------------------------------------------------
\input{../settings/colors}
\usecolortheme[named=bleuUni]{structure}

% Special commands : ddfrac and actionenv and compresslist
\newcommand\ddfrac[2]{\frac{\displaystyle #1}{\displaystyle #2}}

\newenvironment<>{varblock}[2][\textwidth]{%
  \setlength{\textwidth}{#1}
  \begin{actionenv}#3%
    \def\insertblocktitle{#2}%
    \par%
    \usebeamertemplate{block begin}}
  {\par%
    \usebeamertemplate{block end}%
  \end{actionenv}}

\newcommand{\compresslist}{ % Define a command to reduce spacing within itemize/enumerate environments, this is used right after \begin{itemize} or \begin{enumerate}
\setlength{\itemsep}{1pt}
\setlength{\parskip}{0pt}
\setlength{\parsep}{0pt}
}

\newcommand*\circled[1]{\tikz[baseline=(char.base)]{%
      \node[shape=circle,fill=bleuUni,inner sep=2pt] (char) {\textbf{\textcolor{white}{#1}}};}}

\newcommand{\itmsp}[1] {\setlength\itemsep{#1}}
%%%%%%%%%%%%%%%%%%%%%%%%%

%%%%% Beamer
\usepackage[bars]{beamerthemetree} % Beamer theme v 2.2
\mode<presentation>
\newcommand*\oldmacro{}%
\let\oldmacro\insertshorttitle%
\renewcommand*\insertshorttitle{%
 \oldmacro\hspace{0pt plus 1 filll}%
\insertframenumber\,/\,\inserttotalframenumber}
\setbeamertemplate{footline}[frame number]
\setbeamersize{text margin left=10pt,text margin right=10pt}
\setbeamerfont{frametitle}{size=\small}
\setbeamertemplate{frametitle}{ \nointerlineskip %
\begin{beamercolorbox}[wd=\paperwidth,ht=2.2ex,dp=.9ex,left]{frametitle} %
                       \hspace*{2ex}\strut\bfseries\color{bleuUni!15!white}\insertframetitle\strut\par %
\end{beamercolorbox}}

%\setbeamerfont{headline}{size=\footnotesize}

% \usepackage{animate}
% \usepackage{multimedia}
\usetheme{Ilmenau} % Beamer theme v 3.0
\setbeamercolor{section in head/foot}{bg=marronUni}
\useinnertheme{circles} %rectangle bullet points instead of circle ones
\usepackage{beamerthemebars}
\beamertemplatenavigationsymbolsempty
%\setbeamercolor{navigation symbols dimmed}{fg=red!80!black}
%\setbeamercolor{navigation symbols}{fg=red!80!black}
%%%%%%%%%%%%%%%%%%%%%%%%%

\setbeamertemplate{headline}{%
\begin{beamercolorbox}[colsep=1.5pt]{upper separation line head}
\end{beamercolorbox}
\begin{beamercolorbox}{section in head/foot}
    \vskip2pt\insertsectionnavigationhorizontal{\paperwidth}{}{}\vskip2pt
\end{beamercolorbox}%
\begin{beamercolorbox}[ht=10pt]{subsection in head/foot}%
    \vskip2pt\insertsubsectionnavigationhorizontal{\paperwidth}{}{}\vskip2pt
\end{beamercolorbox}%
\begin{beamercolorbox}[colsep=1.5pt]{lower separation line head}
\end{beamercolorbox}
}

\setbeamertemplate{blocks}[rounded][shadow=false]

%%%%% Contents each new sec and subsec
\AtBeginSection[]
{
  %\ifnumcomp{\value{section}}{=}{1}{}{
    \begin{frame}[c]{Plan}
      \centering
      \tableofcontents[
          currentsection,
          hideothersubsections
          %subsectionstyle=show/hide
      ]
    \end{frame}
  %}
}

\AtBeginSubsection[]
{
  \begin{frame}[c]{Plan}
    \tableofcontents[
      currentsection,
      sectionstyle=show/shaded,
      subsectionstyle=show/shaded/hide
    ]
  \end{frame}
}
