%!TEX root = ./slides.tex

% -----------------------------------------------------------------------------
% Packages --------------------------------------------------------------------
% -----------------------------------------------------------------------------

\usepackage[T1]{fontenc}
\usepackage[utf8]{inputenc}
\usepackage[english]{babel} % \usepackage[frenchb]{babel}
\usepackage{eulervm}
\usepackage{etoolbox,refcount}
\usepackage[normalem]{ulem} % strikeout with \sout{}
\usepackage{booktabs}
\usepackage{multirow}
\usepackage{multicol}
\usepackage{makecell}
\usepackage{pgfplots}
\usepackage{tikz}
\usepackage{environ}
\usepackage{circuitikz}
\usepackage{tikz-timing} % package pour les chronogrammes
\usepackage{caption}
\usepackage{graphicx}
\usepackage[export]{adjustbox} % align in includegraphics
\usepackage{stmaryrd} % llbracket
\usepackage{amsmath}
\usepackage{amssymb}
\usepackage{marvosym} % grosse fleche
\usepackage{calc}
\usepackage[english,onelanguage,ruled,linesnumbered,vlined]{algorithm2e}
\usepackage{bm}
\usepackage{pifont}
\usepackage{minted} % beautiful source code
\usepackage{tcolorbox}
\usepackage{etoolbox}
\usepackage{appendixnumberbeamer} % do not count backup slides in the page counter
\usepackage[
  labelnumber  = true,
  backend      = biber,
  sorting      = none,
  % style        = alphabetic,
  % citestyle    = alphabetic,
  maxnames     = 6,
  defernumbers = true,
  isbn         = false,
  doi          = true,
  url          = true,
  backref      = true]{biblatex}
\addbibresource{../tail/bibliography.bib}

% -----------------------------------------------------------------------------
% Configuration of the packages -----------------------------------------------
% -----------------------------------------------------------------------------

% load the colors definition from an external file
\input{../settings/colors}

% smaller footnotesize
\renewcommand{\footnotesize}{\fontsize{7}{8.4}}

% reset the footnote counter to 0 for each slide
\AtBeginEnvironment{frame}{\setcounter{footnote}{0}}

% package 'minted'
% following lines are for a beautiful source code with 'minted'
\setminted{bgcolor=marronUni,fontsize=\fontsize{8}{9.2},tabsize=4,numbers=left,framesep=0pt,numbersep=2pt,xleftmargin=3.5mm,xrightmargin=3.5mm,highlightcolor=marronUni!90} % package 'minted'
\usemintedstyle{monokai}
\BeforeBeginEnvironment{minted}{\begin{tcolorbox}[colback=marronUni, colframe=black, arc=0.5mm, boxsep=0mm, boxrule=0.0mm, left=0mm, right=0mm, top=0mm, bottom=0mm]}
\AfterEndEnvironment{minted}{\end{tcolorbox}}
\renewcommand{\theFancyVerbLine}{\textcolor[rgb]{1,1,1}{\tiny{\arabic{FancyVerbLine}}}}

% package 'tikz'
\usetikzlibrary{matrix, positioning, fit, patterns, shapes, arrows, shapes.multipart, decorations.pathmorphing, calc, spy}
\tikzset{
    invisible/.style={opacity=0},
    visible on/.style={alt={#1{}{invisible}}},
    alt/.code args={<#1>#2#3}{%
      \alt<#1>{\pgfkeysalso{#2}}{\pgfkeysalso{#3}} % \pgfkeysalso doesn't change the path
    },
    hatch distance/.store in=\hatchdistance,
    hatch distance=7pt,
    hatch thickness/.store in=\hatchthickness,
    hatch thickness=0.5pt,
}

% package 'pgfplots'
\pgfplotsset{compat=newest}
\usepgfplotslibrary{colorbrewer}
\usepgfplotslibrary{groupplots}
% new shapes for pgfplot
\makeatletter
\pgfdeclarepatternformonly[\hatchdistance,\hatchthickness]{flexible hatch north east}
{\pgfqpoint{0pt}{0pt}}
{\pgfqpoint{\hatchdistance}{\hatchdistance}}
{\pgfpoint{\hatchdistance-1pt}{\hatchdistance-1pt}}%
{
    \pgfsetcolor{\tikz@pattern@color}
    \pgfsetlinewidth{\hatchthickness}
    \pgfpathmoveto{\pgfqpoint{0pt}{0pt}}
    \pgfpathlineto{\pgfqpoint{\hatchdistance}{\hatchdistance}}
    \pgfusepath{stroke}
}
\makeatletter
\pgfdeclarepatternformonly[\hatchdistance,\hatchthickness]{flexible hatch north west}
{\pgfqpoint{0pt}{0pt}}
{\pgfqpoint{\hatchdistance}{\hatchdistance}}
{\pgfpoint{\hatchdistance-1pt}{\hatchdistance-1pt}}%
{
    \pgfsetcolor{\tikz@pattern@color}
    \pgfsetlinewidth{\hatchthickness}
    \pgfpathmoveto{\pgfqpoint{\hatchdistance}{0pt}}
    \pgfpathlineto{\pgfqpoint{0pt}{\hatchdistance}}
    \pgfusepath{stroke}
}

% configure figure captions
\setbeamertemplate{caption}{\raggedright\insertcaption\par}

% package 'biblatex'
% separate multi-citations by a comma instead of a semicolon in 'biblatex'
\renewcommand*{\multicitedelim}{\addcomma\addspace}
\setbeamertemplate{bibliography item}{\textcolor{black}{\insertbiblabel}}

% package 'hyperref'
\hypersetup{
  colorlinks,
  allcolors=.,
  citecolor= bluecite,
  urlcolor=bluecite,
}

% package 'url'
\urlstyle{tt}

% package 'algorithm2e'
\SetKwComment{Comment}{$\triangleright$\ }{}
\newcommand\mycommfont[1]{\small\ttfamily\textcolor{Comment}{#1}}
\SetCommentSty{mycommfont}

% -----------------------------------------------------------------------------
% Beamer template configuration -----------------------------------------------
% -----------------------------------------------------------------------------

\usecolortheme[named=bleuUni]{structure}

% Special commands : ddfrac and actionenv and compresslist
\newcommand\ddfrac[2]{\frac{\displaystyle #1}{\displaystyle #2}}

\newenvironment<>{varblock}[2][\textwidth]{%
  \setlength{\textwidth}{#1}
  \begin{actionenv}#3%
    \def\insertblocktitle{#2}%
    \par%
    \usebeamertemplate{block begin}}
  {\par%
    \usebeamertemplate{block end}%
  \end{actionenv}}

\newcommand{\compresslist}{ % Define a command to reduce spacing within itemize/enumerate environments, this is used right after \begin{itemize} or \begin{enumerate}
\setlength{\itemsep}{1pt}
\setlength{\parskip}{0pt}
\setlength{\parsep}{0pt}
}

\newcommand*\circled[1]{\tikz[baseline=(char.base)]{%
      \node[shape=circle,fill=bleuUni,inner sep=2pt] (char) {\textbf{\textcolor{white}{#1}}};}}

\newcommand{\itmsp}[1] {\setlength\itemsep{#1}}
%%%%%%%%%%%%%%%%%%%%%%%%%

%%%%% Beamer
\usepackage[bars]{beamerthemetree} % Beamer theme v 2.2
\mode<presentation>
\newcommand*\oldmacro{}%
\let\oldmacro\insertshorttitle%
\renewcommand*\insertshorttitle{%
 \oldmacro\hspace{0pt plus 1 filll}%
\insertframenumber\,/\,\inserttotalframenumber}
\setbeamertemplate{footline}[frame number]
\setbeamersize{text margin left=10pt,text margin right=10pt}
\setbeamerfont{frametitle}{size=\small}
\setbeamertemplate{frametitle}{ \nointerlineskip %
\begin{beamercolorbox}[wd=\paperwidth,ht=2.2ex,dp=.9ex,left]{frametitle} %
                       \hspace*{2ex}\strut\bfseries\color{bleuUni!15!white}\insertframetitle\strut\par %
\end{beamercolorbox}}

\usetheme{Ilmenau} % Beamer theme v 3.0
\setbeamercolor{section in head/foot}{bg=marronUni}
\useinnertheme{circles} %rectangle bullet points instead of circle ones
\usepackage{beamerthemebars}
\beamertemplatenavigationsymbolsempty
%%%%%%%%%%%%%%%%%%%%%%%%%

\setbeamertemplate{headline}{%
\begin{beamercolorbox}[colsep=1.5pt]{upper separation line head}
\end{beamercolorbox}
\begin{beamercolorbox}{section in head/foot}
    \vskip2pt\insertsectionnavigationhorizontal{\paperwidth}{}{}\vskip2pt
\end{beamercolorbox}%
\begin{beamercolorbox}[ht=10pt]{subsection in head/foot}%
    \vskip2pt\insertsubsectionnavigationhorizontal{\paperwidth}{}{}\vskip2pt
\end{beamercolorbox}%
\begin{beamercolorbox}[colsep=1.5pt]{lower separation line head}
\end{beamercolorbox}
}

\setbeamertemplate{blocks}[rounded][shadow=false]

\AtBeginSection[]
{
\begin{frame}[c]{Organization of the Presentation}
  \tableofcontents[
      currentsection,
      subsectionstyle=hide,
  ]
\end{frame}
}

% -----------------------------------------------------------------------------
% New commands declaration ----------------------------------------------------
% -----------------------------------------------------------------------------

\newcommand{\MIPP}{MIPP\xspace}
\newcommand{\longMIPP}{MyIntrinsics++\xspace}
\newcommand{\TSIMD}{T-SIMD\xspace}
\newcommand{\xsimd}{xsimd\xspace}
\newcommand{\simdpp}{simdpp\xspace}
\newcommand{\Vc}{Vc\xspace}
\newcommand{\VCL}{VCL\xspace}
\newcommand{\BoostSIMD}{Boost.SIMD\xspace}
\newcommand{\bSIMD}{bSIMD\xspace}
\newcommand{\AFFECT}{AFF3CT\xspace}
\newcommand{\C}{\texttt{C}\xspace}
\newcommand{\Cxx}{\texttt{C++}\xspace}
\newcommand{\Cxy}[1]{\texttt{C++{#1}}\xspace}
\newcommand{\CppUnit}{\texttt{CppUnit}\xspace}
\newcommand{\cmark}{\textcolor{btfGreen}{\ding{51}}}
\newcommand{\xmark}{\textcolor{btfRed}{\ding{55}}}
\newcommand{\Arikan}{Ar{\i}kan\xspace}
\newcommand{\highlight}[2][yellow]{\mathchoice%
  {{\setlength{\fboxsep}{0pt}\colorbox{#1}{$\displaystyle#2$}}}%
  {{\setlength{\fboxsep}{0pt}\colorbox{#1}{$\textstyle#2$}}}%
  {{\setlength{\fboxsep}{0pt}\colorbox{#1}{$\scriptstyle#2$}}}%
  {{\setlength{\fboxsep}{0pt}\colorbox{#1}{$\scriptscriptstyle#2$}}}}%
\newcommand{\R}{\textsuperscript{\textregistered}\xspace}
\newcommand{\TM}{\textsuperscript{\texttrademark}\xspace}
\newcommand{\enumcite}[1]{{{\fontsize{6}{7.2}\selectfont \cite{#1}\quad \fullcite{#1}}}}

% -----------------------------------------------------------------------------
% New math operators declaration ----------------------------------------------
% -----------------------------------------------------------------------------

\DeclareMathOperator*{\sign}{sign}
\DeclareMathOperator*{\DecoderSC}{DecoderSC}
\DeclareMathOperator*{\hardDecide}{h_d}

% -----------------------------------------------------------------------------
% Add column types for big table ----------------------------------------------
% -----------------------------------------------------------------------------

\newcolumntype{L}[1]{>{\raggedright\let\newline\\\arraybackslash\hspace{0pt}}m{#1}}
\newcolumntype{C}[1]{>{\centering\let\newline\\\arraybackslash\hspace{0pt}}m{#1}}
\newcolumntype{R}[1]{>{\raggedleft\let\newline\\\arraybackslash\hspace{0pt}}m{#1}}
\newlength{\simcolwidth}
\setlength{\simcolwidth}{0.5cm}