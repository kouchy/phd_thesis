%!TEX root = ../my_thesis.tex
\chapter{Efficient Implementations of Receiver Algorithms}

\section{Vectorization~\cite{Cassagne2018}}

\begin{itemize}
  \item \textbf{état de l'art wrapper SIMD}
  \item le temps d'exécution d'une tâche peut varier entre quelques
        microsecondes et quelques milliseconde -> faible latence -> adapté à la
        vectorisation
  \item expliquer les approches intra-trame et inter-trame pour faire du SIMD
\end{itemize}

\section{Channel Coding}

\subsection{Polar Decoders~\cite{Cassagne2015c,Cassagne2016b,Leonardon2019}}

\begin{itemize}
  \item \textbf{état de l'art} (HoF polar: \url{http://aff3ct.github.io/hof_polar.html})
  \item décodeur SC
  \item décodeur SC-List
  \item polar API
  \item déroulage d'arbre
  \item décodeurs générique (pas déroulé)
  \item élagage
  \item inter/intra-SIMD
  \item consommation énergétique ARM/x86
  \item 32-bit, 16-bit, 8-bit
\end{itemize}

\begin{itemize}
  \item \textbf{état de l'art multi-kernel}
  \item factorisation automatique de kernels et génération du code source associé
  \item décodeurs multi-kernels génériques SC, SCL, CA-SCL et ASCL (non systématique et systématique)
\end{itemize}

\subsection{Turbo Decoders~\cite{Cassagne2016a}}

\begin{itemize}
  \item \textbf{état de l'art} (HoF turbo: \url{http://aff3ct.github.io/hof_turbo.html})
  \item décodeur Turbo LTE (3G/4G)
  \item inter-SIMD
  \item treillis
  \item décodeur inter- et intra-SIMD LTE à faible latence
  \item 32-bit, 16-bit, 8-bit
\end{itemize}

\subsection{LDPC Decoders}

\begin{itemize}
  \item \textbf{état de l'art} (HoF LDPC: \url{http://aff3ct.github.io/hof_ldpc.html})
  \item décodeurs génériques (BP-flooding/HL/VL) sur les "update nodes"
  \item versions séquentielles et inter-SIMD
  \item 32-bit, 16-bit
  \item graphe biparti
\end{itemize}

\section{Modem}

\subsection{SCMA Demodulators~\cite{Ghaffari2019}}

\begin{itemize}
  \item \textbf{état de l'art SCMA}
  \item démodulateur SCMA intra-SIMD
  \item proposition d'une approximation des calculs
\end{itemize}
